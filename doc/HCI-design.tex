\documentclass{article} \usepackage{CJK}
\usepackage{CJK}
\usepackage{ctex}
\usepackage{graphicx}
\renewcommand{\contentsname}{目录}
\renewcommand{\abstractname}{摘要}
\author{杨铭 - 5130379022\\
        李晟 - 5130379017\\
        张云翔 - 51303790XX}
\title{HCI 课程选题提交}
\begin{document}
\maketitle
\section{基于Leap Motion的手势3D建模}
\newpage
\section{基于Leap Motion的手势控制魔方音游}
\textbf{杨铭 - 5130379022 \& 李晟 - 5130379017}
\subsection{用户分析}
音游作为较为小众的游戏,受众也不是很大。我们这个想法本质还是一个音游,所以用户群体为广大游戏爱好者特别是音游爱好者。
\subsection{目标}
做出有一定游戏性的通过识别手势进行输入的一个音乐游戏。可能包含以下特性:
\begin{itemize}
  \item 几个自带曲目,可能会有铺面编辑器功能。
  \item 多种基本手势输入法,比如手指滑动,手掌翻动等
  \item 游戏主体是一个魔方,玩家需要按照游戏内的提示给出相应手势来完成合乎音乐节拍的“打击”
\end{itemize}
\subsection{创新点}
使用手势操作,革新了传统音乐游戏的玩法。
\subsection{技术支持}
\begin{itemize}
  \item Leap Motion硬件支持
  \item 手势识别
\end{itemize}
\newpage
\section{}


\end{document}
